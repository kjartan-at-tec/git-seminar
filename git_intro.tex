% Created 2020-08-28 vie 17:04
% Intended LaTeX compiler: pdflatex
\documentclass[presentation,aspectratio=1610]{beamer}
\usepackage[utf8]{inputenc}
\usepackage[T1]{fontenc}
\usepackage{graphicx}
\usepackage{grffile}
\usepackage{longtable}
\usepackage{wrapfig}
\usepackage{rotating}
\usepackage[normalem]{ulem}
\usepackage{amsmath}
\usepackage{textcomp}
\usepackage{amssymb}
\usepackage{capt-of}
\usepackage{hyperref}
\usepackage{khpreamble}
\usetheme{default}
\author{Kjartan Halvorsen}
\date{2020-08-29}
\title{git and GitHub an quick introduction}
\hypersetup{
 pdfauthor={Kjartan Halvorsen},
 pdftitle={git and GitHub an quick introduction},
 pdfkeywords={},
 pdfsubject={},
 pdfcreator={Emacs 26.3 (Org mode 9.3.6)}, 
 pdflang={English}}
\begin{document}

\maketitle

\section{Intro}
\label{sec:org55ce358}
\begin{frame}[label={sec:org6f0d367}]{Resources}
\url{https://git-scm.com/book/en/v2/}
\url{https://www.atlassian.com/git/tutorials/}
\url{https://www.atlassian.com/git/tutorials/atlassian-git-cheatsheet}
\end{frame}

\begin{frame}[label={sec:orge821001}]{What is git?}
A version control system
\end{frame}

\begin{frame}[label={sec:org6888b33}]{What is version control?}
The \alert{correct} solution to a common problem:

       \begin{center}
       \begin{tikzpicture}
         \node (linus) at (0,0) {\includegraphics[width=5cm]{figures/Linus.jpg}};
          \node [draw, align=center,
           cloud callout, cloud puffs = 17, cloud puff arc=140,
           callout pointer segments = 3, anchor = pointer,
callout relative pointer = {(330:2cm)},
           aspect = 3, ] at (-3, 1.6)
      {Oh no!\\My code doesn't work anymore!\\It worked perfectly last Monday!};
       \end{tikzpicture}
       \end{center}
\end{frame}


\begin{frame}[label={sec:org3b29ae9}]{What is GitHub?}
An online service to host \alert{git} repositories (and a with a lot of services to manage code projects)

It is not the only option. There's also \href{https://about.gitlab.com/}{GitLab} and \href{https://bitbucket.org/product}{Bitbucket}, and others.
\end{frame}

\begin{frame}[label={sec:org49fca42}]{Version control the git way}
\begin{center}
\includegraphics[width=0.8\linewidth]{figures/checkins.png}
\end{center}
\end{frame}

\begin{frame}[label={sec:org60d87f7}]{Version control the git way}
It is \alert{distributed}!

\begin{center}
\includegraphics[width=0.5\linewidth]{figures/distributed.png}
\end{center}
\end{frame}


\begin{frame}[label={sec:org64bc9ec}]{Let's start locally}
The three states in your local \alert{git} repository

\begin{center}
\includegraphics[width=0.8\linewidth]{figures/three-states.png}
\end{center}
\end{frame}
\begin{frame}[label={sec:org737575c}]{Let's start locally}
The life of a file in \alert{git}

\begin{center}
\includegraphics[width=0.8\linewidth]{figures/lifecycle-files.png}
\end{center}
\end{frame}
\begin{frame}[label={sec:org61e3195}]{Let's start locally}
\begin{block}{Goal: Understand local workflow}
\begin{enumerate}
\item Create a repository
\item Add and commit some files
\item Create a branch, then merge
\end{enumerate}
\end{block}
\end{frame}


\begin{frame}[label={sec:orgc6ada8a}]{Let's go online}
\begin{block}{Goal: Understand online/collaborative workflow}
\begin{enumerate}
\item Clone a repository
\item Create a branch to work on
\item Pull in upstream changes
\item Merge/rebase
\item Push to online repository
\end{enumerate}
\end{block}
\end{frame}

\begin{frame}[label={sec:orgff70be7}]{(Re)sources}
\begin{itemize}
\item \url{https://git-scm.com/book/en/v2/}
\item \url{https://www.atlassian.com/git/tutorials/atlassian-git-cheatsheet}
\item \url{https://github.com/k88hudson/git-flight-rules}
\end{itemize}
\end{frame}
\end{document}