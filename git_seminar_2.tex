% Created 2019-08-26 lun 15:15
\documentclass[presentation,aspectratio=1610]{beamer}
\usepackage[utf8]{inputenc}
\usepackage[T1]{fontenc}
\usepackage{fixltx2e}
\usepackage{graphicx}
\usepackage{longtable}
\usepackage{float}
\usepackage{wrapfig}
\usepackage{rotating}
\usepackage[normalem]{ulem}
\usepackage{amsmath}
\usepackage{textcomp}
\usepackage{marvosym}
\usepackage{wasysym}
\usepackage{amssymb}
\usepackage{hyperref}
\tolerance=1000
\usepackage{khpreamble}
\usetheme{default}
\author{Kjartan Halvorsen}
\date{}
\title{git and GitHub - Collaboration and workflow}
\hypersetup{
  pdfkeywords={},
  pdfsubject={},
  pdfcreator={Emacs 25.3.50.2 (Org mode 8.2.10)}}
\begin{document}

\maketitle


\section{Objective}
\label{sec-1}
\begin{frame}[label=sec-1-1]{Goal}
\alert{Understanding how to collaborate with git and GitHub}
\end{frame}

\section{Resources}
\label{sec-2}
\begin{frame}[label=sec-2-1]{(Re)sources}
\begin{itemize}
\item \url{https://git-scm.com/book/en/v2/}
\item \url{https://www.atlassian.com/git/tutorials/atlassian-git-cheatsheet}
\item \url{https://desktop.github.com}
\end{itemize}
\end{frame}

\section{Repetition}
\label{sec-3}
\begin{frame}[label=sec-3-1]{Version control the git way}
\begin{center}
\includegraphics[width=0.8\linewidth]{figures/checkins.png}
\end{center}
\end{frame}

\begin{frame}[label=sec-3-2]{Version control the git way}
It is \alert{distributed}!

\begin{center}
\includegraphics[width=0.5\linewidth]{figures/distributed.png}
\end{center}
\end{frame}

\begin{frame}[fragile,label=sec-3-3]{Branch, commit, merge and rebase}
 \begin{itemize}
\item To create a branch and checking it out immediately:
\texttt{git checkout -b newFeature}
Next commit will be on the newFeature branch
\item To merge commits done on newFeature into the master branch:
\texttt{git checkout master; git merge newFeature}
This creates a new commit containing the merge
\item To move the commits done on newFeature onto the master branch
\texttt{git rebase master}
This is  done while on newFeature
\end{itemize}
\end{frame}


\section{Collaboration}
\label{sec-4}

\begin{frame}[label=sec-4-1]{A useful model for collaboration}
\url{https://nvie.com/files/Git-branching-model.pdf}

From \url{https://nvie.com/posts/a-successful-git-branching-model/}
\end{frame}

\begin{frame}[fragile,label=sec-4-2]{Principles}
 \begin{itemize}
\item Keep the \texttt{master} branch production-ready at \alert{all times}
\item Do all development in branches
\item Use \alert{pull requests} to handle suggested changes to the \texttt{master} branch
\item Two models: \alert{Fork and Pull} and \alert{Shared Repository}
\begin{description}
\item[{Fork and pull}] Contributors fork the repository, make changes and make pull requests
\item[{Shared repository}] Contributors have push rights to original repository
\end{description}
\end{itemize}
\end{frame}

\begin{frame}[fragile,label=sec-4-3]{Fork clone and add upstream}
 After forking and cloning:
\begin{verbatim}
git remote add upstream https://github.com/kjartan-at-tec/git-seminar.git
\end{verbatim}
\end{frame}

\begin{frame}[fragile,label=sec-4-4]{Branch and add changes}
 In your cloned fork, create a new branch
\begin{verbatim}
git checkout -b my-contributions
\end{verbatim}

Modify \texttt{README.md},  add to staging area, commit.
\end{frame}

\begin{frame}[fragile,label=sec-4-5]{Submit a pull request}
 \begin{enumerate}
\item Make your local master branch up-to-date with origin and upstream
\begin{enumerate}
\item Checkout master \texttt{git checkout master}
\item Fetch changes from upstream repository \texttt{git pull}
\item Checkout development branch \texttt{git checkout my-contributions}
\item Merge master branch \texttt{git merge master}
\end{enumerate}
\item Push branch to origin \texttt{git push origin my-contributions}
\item On GitHub, create a pull request.
\end{enumerate}
\end{frame}
% Emacs 25.3.50.2 (Org mode 8.2.10)
\end{document}